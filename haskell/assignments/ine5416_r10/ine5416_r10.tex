\documentclass{article}

\usepackage[utf8]{inputenc}
\usepackage[a4paper, left=20mm, right=20mm, top=20mm, bottom=20mm]{geometry}

\begin{document}

\subsubsection*{INE5416 - Paradigmas de Programação (2015/2) \\
    Gustavo Zambonin \\
    Relatório 10 - Mônadas
}

\textbf{Nota}: todos os excertos de código foram executados com
\texttt{ghci ine5416\_r10.hs} e chamados no interpretador.

\subsubsection*{Parte 1}
\begin{itemize}
    \item O conceito de homologia algébrica envolve outras definições abstratas
    utilizadas para construir este estudo. Um espaço topológico pode ser
    definido como um conjunto de pontos e suas vizinhanças, que satisfazem um
    conjunto de axiomas relacionando estes. Permitem formalizar conceitos como
    convergência e continuidade. A homologia algébrica é o ramo da matemática
    que estuda estes espaços em um ambiente de caráter algébrico.

    \item Um functor, em teoria das categorias (ramo da matemática que trata
    de formas abstrata estruturas matemáticas), é um mapeamento entre
    categorias que preserva estruturas. Pode-se pensar que um functor é um
    homomorfismo entre categorias.

    \item As mônadas, em programação funcional, são estruturas que representam
    computações definidas como sequências de passos (operações em cadeia, por
    exemplo), permitindo ao programador processar a informação passo a passo,
    adicionando mais regras ao processamento quando necessário. Um functor
    porta-se, de grosso modo, como uma generalização de métodos de mapeamento
    a quaisquer tipos passados como argumento.
\end{itemize}

\subsubsection*{Parte 2}

É possível perceber claramente o uso de uma abordagem monádica no código
produzido em Haskell, embora ainda mostre-se necessário o uso de estruturas
condicionais em uma parte crítica do código. Esquiva-se de uma divisão com zero
verificando se algum dos termos é zero, retornando o tipo \texttt{Nothing}; a
definição explícita de uma função soma segue logo abaixo; e por fim, a
resistência de dois resistores é obtida aplicando, em cadeia, todas as mônadas
implementadas. O tratamento de divisões por zero, neste caso, torna-se muito
mais intuitivo, protegendo o código de tais erros.

\end{document}
