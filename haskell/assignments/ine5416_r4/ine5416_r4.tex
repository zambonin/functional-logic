\documentclass{article}

\usepackage[utf8]{inputenc}
\usepackage[a4paper, left=20mm, right=20mm, top=20mm, bottom=20mm]{geometry}
\usepackage[colorlinks=true, urlcolor=blue]{hyperref}
\usepackage{amsfonts}
\usepackage{mathtools}

\begin{document}

\subsubsection*{INE5416 - Paradigmas de Programação (2015/2) \\
    Gustavo Zambonin \\
    Relatório 4 - Cálculo-$\lambda$
}

\subsubsection*{Parte 1}
\begin{itemize}
    \item \textit{An unsolvable problem of elementary number theory} pode ser
    obtido
    \href{http://www.ics.uci.edu/~lopes/teaching/inf212W12/readings/church.pdf}
    {aqui}, enquanto existem apenas excertos de \textit{The Calculi of
    Lambda-Conversion} na internet.

    \item A principal diferença entre uma variável, em funções matemáticas, e um
    argumento, no cálculo-$\lambda$, é a limitação em relação ao domínio da
    função. Na sintaxe usual, tem-se uma função $f(x)$ qualquer onde apenas
    elementos que respeitam a $x$ devem ser manipulados. De outro modo,
    argumentos podem ser usados livremente para modificar a expressão no
    cálculo-$\lambda$.

    \item Abaixo segue um pequeno resumo sobre os três tipos de reduções no
    cálculo-$\lambda$.
    \begin{itemize}
        \item \textbf{conversão-$\alpha$} (ou conversão alfa): responsável por
        renomear variáveis se assim for necessário para o escopo da expressão.
        Por exemplo: $\lambda x.x \stackrel{\alpha}{\rightarrow} \lambda y.y$.

        \item \textbf{redução-$\beta$} (ou redução beta): a mais comum das
        operações de redução por uma grande margem, habilita o processo de
        calcular um resultado da aplicação de uma função a uma expressão. Por
        exemplo: $\lambda x.x + y\ (7) \stackrel{\beta}{\rightarrow} 7 + y$.

        \item \textbf{conversão-$\eta$} (ou conversão eta): elimina redundâncias
        nas abstrações, no caso de uma função ser utilizada apenas para passar
        seu argumento a outras expressões. Por exemplo: $\lambda x.Mx
        \stackrel{\eta}{\rightarrow} M$, onde $x$ não pode ser uma variável
        livre em $M$.
    \end{itemize}
\end{itemize}

\subsubsection*{Parte 2}
Utilizando o interpretador Haskell online \href{http://ghc.io}{ghc.io}, as
resoluções para os problemas apresentados seguem abaixo. O módulo
\texttt{Data.List} precisou ser importado para utilização das funções sugeridas.
\begin{itemize}
    \item Utilizando aritmética modular simples, deleta-se o primeiro elemento
    da lista sugerida que responde ao requerimento necessário (divisibilidade
    por três).
    \begin{verbatim}Prelude> deleteBy(\x y -> y `mod` x == 0) 3 [5..10]
[5,7,8,9,10]\end{verbatim}

    \item De modo semelhante à estratégia anterior, mas agora filtrando os
    elementos que respeitam à regra imposta, tem-se a segunda resolução.
    \begin{verbatim}Prelude> filter(\x -> x `mod` 4 == 0) [4..19]
[4,8,12,16]\end{verbatim}

    \item O resultado da expressão apresentada segue abaixo.
    \begin{verbatim}Prelude> [x | x <- [1..4], y <- [x..5], (x+y) `mod` 2 == 0]
[1,1,1,2,2,3,3,4]\end{verbatim}
\end{itemize}

\newpage

\subsubsection*{Lista de exercícios 1}
\begin{enumerate}\bfseries
    \item $y = \frac{x+1}{x^2}$ \\
    $f(y)=$ ?
        \begin{gather*}
            yx^2 = x + 1 \\
            yx^2 - x - 1 = 0 \\
            x = \frac{-(-1) \pm \sqrt{(-1)^2 - 4y(-1)}}{2y} \\
            \mathbf{x = \frac{1 \pm \sqrt{4y + 1}}{2y}, y \neq 0}
        \end{gather*}

    \item $f(x-1) = x^2 - 1$ \\
    $f(x)=$ ?
        \begin{gather*}
            f((x + 1) - 1) = (x + 1)^2 - 1 \\
            f(x) = x^2 + 2x + 1 - 1 \\
            \mathbf{f(x) = x^2 + 2x}
        \end{gather*}

    \item $f(x) = x + \frac{1}{x}$ \\
    $(f(x))^3 \stackrel{?}{=} f(x^3) + 3f(\frac{1}{x})$
        \begin{gather*}
            (f(x))^3 = (x^3 + \frac{1}{x^3}) + 3({\frac{1}{x}} +
            \frac{1}{\frac{1}{x}}) \\
            (f(x))^3 = (x^3 + \frac{1}{x^3}) + 3(\frac{1}{x} + x) \\
            (x + \frac{1}{x})^3 = x^3 + \frac{1}{x^3} + {\frac{3}{x} + 3x} \\
            x^3 + \frac{3x^2}{x} + \frac{3x}{x^2} + \frac{1}{x^3} = x^3 +
            \frac{1}{x^3} + {\frac{3}{x} + 3x} \\
           \mathbf{x^3 + 3x + \frac{3}{x} + \frac{1}{x^3} = x^3 + \frac{1}{x^3}
           + {\frac{3}{x}} + 3x}
        \end{gather*}

    \item $f(x) = \frac{|x|}{x}$, $x \neq 0$ \\
    $|f(a) - f(-a)| =$ ?
        \begin{gather*}
            \frac{|a|}{a} - \frac{|-a|}{-a} =
            \frac{|a|}{a} + \frac{|a|}{a} = 2\frac{|a|}{a} = \mathbf{\pm2}
        \end{gather*}

    \item $f(x) = x^4$, $g(x) = \sqrt{1+x^3}$, $h(x) = \frac{x^2 + 1}{2x + 1}$;
    $f,g,h : \mathbb{R} \rightarrow [0, \infty)$

    \begin{enumerate}
        \item $f \circ g$ \\ $i(x) = (\sqrt{1+x^3})^4 = (x^3 + 1)^2$; $i :
        \mathbb{R} \rightarrow [0, \infty)$

        \item $f \circ g \circ h$ \\ $j(x) =
        \left(\sqrt{1 +(\frac{x^2 + 1}{2x + 1})^3}\right)^4 =
        \left(1 + (\frac{x^2 + 1}{2x + 1})^3\right)^2$; $j :
        \mathbb{R} \rightarrow [0, \infty)$
    \end{enumerate}
\end{enumerate}

\newpage
\subsubsection*{Lista de exercícios 2}
\begin{enumerate}\bfseries
    \item (a) $f(x) = x^2 + 4$ \\
    $\mathbf{\lambda x.x^2 + 4\ (x)}$ \\ \\
    (b) $f(x) = \sum_{x=1}^{x=10} x$ \\
    $\mathbf{\lambda x.x{1..10}\  (x)}$ \\ \\
    (c) $f(a, b) = a + b$ \\
    $\mathbf{\lambda a.\lambda b.(a+b)}$ \\ \\
    (d) $f(x) = x.x^{-1}$ \\
    $\mathbf{\lambda x.1\ (x)}$

    \item (a) $\lambda x.(\lambda y.y^2 - (\lambda z.(z+x)4)3)2$ \\
    $\lambda x.(\lambda y.y^2 - (4+x)3)2$ \\
    $\lambda x.(3^2 - (4+x))2$ \\
    $9 - 4 - 2 = \mathbf{3}$ \\ \\
    (b) $\lambda x.x + (\lambda y.y^2(b))(a)$ \\
    $\lambda x.x + b^2 (a)$ \\
    $\mathbf{a + b^2}$ \\ \\
    (c) $\lambda x.(\lambda y.(x + (\lambda x.8) - y)6)5$ \\
    $\lambda x.(\lambda y.(x + 8 - y)6)5$ \\
    $\lambda x.(x + 8 - 6)5$ \\
    $5 + 8 - 6 = \mathbf{7}$ \\ \\
    (d) $\lambda xy.x + y\ (3)(7)$ \\
    $3 + 7 = \mathbf{10}$

    \item (a) $\lambda x.x(xy)(\lambda u.u)$ \\
    $\lambda u.u(\lambda u.uy) \\
    \mathbf{\lambda u.uy}$ \\ \\
    (b) $\lambda y.(\lambda x.y \times y + x)(z)$ \\
    $\mathbf{\lambda x.z^2 + x}$ \\ \\
    (c) $\lambda x.(\lambda y.(yx)\lambda i.i)\lambda p.\lambda q.p$ \\
    $\lambda x.(\lambda i.i x)\lambda p.\lambda q.p$ \\
    $\lambda i.i(\lambda p.\lambda q.p)$ \\
    $\mathbf{\lambda p.\lambda q.p}$ \\ \\
    (d) $\lambda x.x(\lambda y.(\lambda x.xy)x)$ \\
    $\lambda x.x(\lambda x.xx)$ \\
    $\mathbf{\lambda x.xx}$ \\ \\
    (e) $(\lambda x.xx)(\lambda y.y)$ \\
    $\lambda y.y(\lambda y.y)$ \\
    $\mathbf{\lambda y.y}$
\end{enumerate}

\end{document}
