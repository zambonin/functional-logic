\documentclass{article}

\usepackage[utf8]{inputenc}
\usepackage[a4paper, left=20mm, right=20mm, top=20mm, bottom=20mm]{geometry}
\usepackage[colorlinks=true, urlcolor=blue]{hyperref}
\usepackage{amsfonts}
\usepackage{mathtools}

\begin{document}

\subsubsection*{INE5416 - Paradigmas de Programação (2015/2) \\
    Gustavo Zambonin \\
    Relatório 9 - Listas em Haskell
}

\textbf{Nota}: todos os excertos de código foram executados com
\texttt{ghci ine5416\_r9.hs} e chamados no interpretador.

\subsubsection*{Parte 1}
\begin{itemize}
    \item Classifica-se uma linguagem como \textit{lazy}, ou ``preguiçosa'',
    quando esta age de modo a calcular uma expressão apenas quando necessário,
    não a cada vez que esta aparece ou é chamada em um código-fonte, evitando
    computações desnecessárias. Esta abordagem permite a potencial construção
    de estruturas infinitas de dados (por exemplo, \texttt{[1..]} em Haskell
    gerará uma lista equivalente a $\mathbb{N^{*}}$ até que não exista mais
    memória disponível para alocação). Em linguagens funcionais,
    \href{http://www.cs.kent.ac.uk/people/staff/dat/miranda/whyfp90.pdf}
    {é prático construir um programa que tenha um módulo gerador de possíveis
    respostas, e um módulo para selecionar uma resposta apropriada} [pg. 9],
    assim necessitando apenas modificar o seletor de respostas, e nunca o
    gerador (que mesmo assim apenas computará o necessário).

    \item O uso de um mapeamento facilita a aplicação de certos critérios em
    um conjunto de elementos, como uma lista. As funções \texttt{map},
    \texttt{reduce} e \texttt{filter} estão disponíveis por padrão em um
    interpretador para Python, e retornam novas listas, criadas a partir do
    critério (geralmente uma operação \texttt{lambda}, mas não se resumindo
    apenas a esta estratégia).

    \begin{itemize}
        \item \texttt{list(map(chr, range(97, 123)))} retorna uma lista com os
        caracteres ASCII $97_{10}$ (\texttt{a} minúsculo) a $122_{10}$
        (\texttt{z} minúsculo).

        \item \texttt{list(map(lambda x: x**2, [1, 2, 3]))} retorna uma lista
        com os quadrados dos números na lista original.
    \end{itemize}

    \item O módulo \texttt{Data.List} é composto de diversos utilitários para
    manipulação de listas, como um operador para concatenação destas, testes
    para lista vazia, retorno de tamanho de lista, funções para reversão,
    mapeamento, retorno de subsequências e permutações, entre outros.

\end{itemize}

\subsubsection*{Parte 2}
\begin{itemize}
    \item A implementação da soma de termos de uma progressão aritmética
    torna-se trivial quando Haskell oferece uma função \texttt{sum} que aceita
    listas como argumento. Desviando desta abordagem, implementa-se a fórmula
    genérica para soma de termos de uma PA da seguinte maneira:
    $S_n = \frac{n(2a_1 + (n - 1)r)}{2}$ é equivalente a
    \begin{verbatim}
    diff n = n!!1 - head(n)
    sumAP n = (length n)*(2*head(n)+(length n - 1)*diff n) `div` 2\end{verbatim}
    Assume-se que a razão da PA é obtida por $a_2 - a_1$, como acontece na
    inferência do Haskell para construção de uma lista com tal característica.

    \item De modo similar, é possível obter o produto de todos os elementos de
    uma lista com a função \texttt{product}. Entretanto, desconsiderando esta
    solução prática, pode-se considerar a fórmula genérica para produto de
    termos de uma PA utilizando a
    \href{http://mathworld.wolfram.com/GammaFunction.html}{função gama}:
    \begin{gather*}
    \Gamma(n) = (n-1)! = \int\limits_0^\infty x^{n-1}e^{-x} \qquad
    \prod\limits_{i=1}^n a_{i} = d^n \cfrac{\Gamma(a_1/d + n)}{\Gamma(a_1/d)}
    \end{gather*}
    Utilizando a \href{https://en.wikipedia.org/wiki/Stirling%27s_approximation}
    {aproximação de Stirling} para fatorial, calcula-se o valor do produto com
    a fórmula a seguir:
    \begin{gather*}
    \Gamma(x) \approx \sqrt{\dfrac{2\pi}{x}}\left(\dfrac{1}{e}\left(x +
    \dfrac{1}{12x - 1/10x}\right)\right)^x
    \end{gather*}
    É válido notar que, por conta de aproximação de ponto flutuante, a resposta
    torna-se discrepante à medida em que o número de termos da PA aumenta. A
    constante $e$ é calculada a partir da série infinita
    $\sum\limits_{n=0}^{\infty} \frac{1}{n!}$.
    \begin{verbatim}
    e = 1 + sum([1 / product[1..x] | x <- take 1000 [1..]])
    gamma n = sqrt(2*pi / n) * (1/e * (n + (1/(12*n - 1/(10*n))))) ** n
    productAP n = (diff n ** fromIntegral (length n)) *
        (gamma ((head(n) / diff n) + fromIntegral (length n)) /
        gamma (head(n) / diff n))
     \end{verbatim}
\end{itemize}

\end{document}
